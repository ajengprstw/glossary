\PassOptionsToPackage{unicode=true}{hyperref} % options for packages loaded elsewhere
\PassOptionsToPackage{hyphens}{url}
%
\documentclass[
]{article}
\usepackage{lmodern}
\usepackage{amssymb,amsmath}
\usepackage{ifxetex,ifluatex}
\ifnum 0\ifxetex 1\fi\ifluatex 1\fi=0 % if pdftex
  \usepackage[T1]{fontenc}
  \usepackage[utf8]{inputenc}
  \usepackage{textcomp} % provides euro and other symbols
\else % if luatex or xelatex
  \usepackage{unicode-math}
  \defaultfontfeatures{Scale=MatchLowercase}
  \defaultfontfeatures[\rmfamily]{Ligatures=TeX,Scale=1}
\fi
% use upquote if available, for straight quotes in verbatim environments
\IfFileExists{upquote.sty}{\usepackage{upquote}}{}
\IfFileExists{microtype.sty}{% use microtype if available
  \usepackage[]{microtype}
  \UseMicrotypeSet[protrusion]{basicmath} % disable protrusion for tt fonts
}{}
\makeatletter
\@ifundefined{KOMAClassName}{% if non-KOMA class
  \IfFileExists{parskip.sty}{%
    \usepackage{parskip}
  }{% else
    \setlength{\parindent}{0pt}
    \setlength{\parskip}{6pt plus 2pt minus 1pt}}
}{% if KOMA class
  \KOMAoptions{parskip=half}}
\makeatother
\usepackage{xcolor}
\IfFileExists{xurl.sty}{\usepackage{xurl}}{} % add URL line breaks if available
\IfFileExists{bookmark.sty}{\usepackage{bookmark}}{\usepackage{hyperref}}
\hypersetup{
  pdfborder={0 0 0},
  breaklinks=true}
\urlstyle{same}  % don't use monospace font for urls
\usepackage[margin=1in]{geometry}
\usepackage{graphicx,grffile}
\makeatletter
\def\maxwidth{\ifdim\Gin@nat@width>\linewidth\linewidth\else\Gin@nat@width\fi}
\def\maxheight{\ifdim\Gin@nat@height>\textheight\textheight\else\Gin@nat@height\fi}
\makeatother
% Scale images if necessary, so that they will not overflow the page
% margins by default, and it is still possible to overwrite the defaults
% using explicit options in \includegraphics[width, height, ...]{}
\setkeys{Gin}{width=\maxwidth,height=\maxheight,keepaspectratio}
\setlength{\emergencystretch}{3em}  % prevent overfull lines
\providecommand{\tightlist}{%
  \setlength{\itemsep}{0pt}\setlength{\parskip}{0pt}}
\setcounter{secnumdepth}{-2}
% Redefines (sub)paragraphs to behave more like sections
\ifx\paragraph\undefined\else
  \let\oldparagraph\paragraph
  \renewcommand{\paragraph}[1]{\oldparagraph{#1}\mbox{}}
\fi
\ifx\subparagraph\undefined\else
  \let\oldsubparagraph\subparagraph
  \renewcommand{\subparagraph}[1]{\oldsubparagraph{#1}\mbox{}}
\fi

% set default figure placement to htbp
\makeatletter
\def\fps@figure{htbp}
\makeatother


\author{}
\date{\vspace{-2.5em}}

\begin{document}

\hypertarget{neural-network}{%
\section{Neural Network}\label{neural-network}}

\begin{itemize}
\item
  \textbf{\emph{Target variable }}

  Variable yang nilainya dipengaruhi oleh \texttt{predictor} dan akan
  diprediksi nilainya, sering disebut sebagai respon/dependent variable.
\item
  \textbf{\emph{Predictor}}

  Variable yang mempengaruhi nilai \texttt{target\ variable} dan
  digunakan untuk memprediksi tersebut, sering disebut sebagi
  independent variable.
\item
  \textbf{\emph{Feature engineering}}

  Menambahkan informasi (variabel/kolom) berdasarkan informasi dari
  variabel lain yang sudah ada.
\item
  \textbf{\emph{Missing value}}

  Keadaan data memiliki nilai yang hilang (tidak diketahui nilainya).
\item
  \textbf{\emph{Standarization}}

  Proses untuk menyeragamkan skala data yang berbeda, sering disebut
  sebagai \texttt{scaling}.
\item
  \textbf{\emph{Data train}}

  Bagian data yang digunakan untuk membuat model (training model).
\item
  \textbf{\emph{Data test}}

  Bagian data yang digunakan untuk mengevaluasi performa model (testing
  model).
\item
  \textbf{\emph{Cross Validation}}

  Membagi data lengkap/utuh menjadi dua bagian data, yaitu
  \texttt{data\ train} dan \texttt{data\ test}.
\item
  \textbf{\emph{Class imbalance}}

  Keadaan dimana suatu kategori/level lebih mendominasi keseluruhan
  \texttt{target\ variable} (kelas mayoritas) dibandingkan
  kategori/level lainnya (kelas minoritas).
\item
  \textbf{\emph{Sampling}}

  Mengambil sebanyak n bagian data secara acak.
\item
  \textbf{\emph{Down-sample}}

  Proses \texttt{sampling} pada observasi kelas mayoritas, sebanyak
  jumlah observasi pada kelas minoritas. Tujuannya untuk menyamakan
  jumlah observasi pada kelas mayoritas dan minoritas.
\item
  \textbf{\emph{Up-sample}}

  Proses \texttt{sampling} pada observasi kelas minoritas, sebanyak
  jumlah observasi pada kelas mayoritas. Tujuannya untuk menyamakan
  jumlah observasi pada kelas mayoritas dan minoritas.
\item
  \textbf{\emph{Optimization}}

  Proses mengoptimumkan suatu nilai dengan menggunakan fungsi turunan,
  pada model neural network meminimumkan nilai error/kesalahan.
\item
  \textbf{\emph{Nodes}}

  Unit terkecil pada arsitektur neural network yang berfungsi untuk
  mengekstrak informasi (feature extraction) dan meneruskan informasi
  tersebut, sering disebut sebagai \texttt{neuron}.
\item
  \textbf{\emph{Input layer}}

  Lapisan pertama pada arsitektur neural network yang terdiri dari
  kumpulan \texttt{nodes}. Jumlah \texttt{nodes} pada
  \texttt{input\ layer} bergantung pada jumlah \texttt{predictor} pada
  data.
\item
  \textbf{\emph{Output layer}}

  Lapisan terakhir pada arsitektur neural network yang terdiri dari
  sebuah \texttt{nodes} atau beberapa \texttt{nodes} bergantung pada
  jenis \texttt{target\ variable}.
\item
  \textbf{\emph{Hidden layer}}

  Lapisan yang terletak di antara \texttt{input\ layer} dan
  \texttt{output\ layer}. Jumlah \texttt{hidden\ layer} dan jumlah
  \texttt{nodes} di setiap \texttt{hidden\ layer} ditentukan oleh
  peneliti.
\item
  \textbf{\emph{Weight}}

  Besar bobot yang menggambarkan besar informasi yang diteruskan dari
  setiap \texttt{nodes}. \texttt{Weight} ditetapkan secara acak.
\item
  \textbf{\emph{Linear regression}}

  Salah satu metode machine learning yang digunakan untuk memprediksi
  \texttt{target\ variable} bertipe numerik/angka.
\item
  \textbf{\emph{Bias}}

  Pada \texttt{linear\ regression} sama seperti nilai
  \texttt{intercept\ (b0)}.
\item
  \textbf{\emph{Activation function}}

  Fungsi yang digunakan untuk mengubah interval nilai informasi yang
  masuk ke setiap \texttt{nodes} pada \texttt{hidden\ layer} dan
  \texttt{output\ layer}.
\item
  \textbf{\emph{Cost function}}

  Fungsi error yang digunakan pada model neural network.
\item
  \textbf{\emph{Feedforward}}

  Proses pada neural network yang dimulai dari \texttt{input\ layer}
  hingga menghasilkan nilai prediksi pada \texttt{output\ layer}.
\item
  \textbf{\emph{Backpropagation}}

  Proses pada neural network ketika melakukan \texttt{optimization} dan
  melakukan update \texttt{weight}.
\item
  \textbf{\emph{Epoch}}

  Satu kali proses \texttt{feedforward} dan \texttt{backpropagation}.
\item
  \textbf{\emph{Gradient}}

  Hasil turunan dari fungsi error/\texttt{cost\ function}.
\item
  \textbf{\emph{Dummy Variable}}

  Hasil transformasi variabel kategorik dengan nilai 0 atau 1. Variabel
  ini digunakan untuk membuat data kategorik yang bersifat kualitatif
  menjadi kuantitatif.
\item
  \textbf{\emph{Learning rate}}

  Besar nilai yang menentukan seberapa cepat proses update
  \texttt{weight} hingga diperoleh nilai erorr yang konstan.
\item
  \textbf{\emph{Batch size}}

  Jumlah observasi yang diikutsertakan untuk satu iterasi.
\item
  \textbf{\emph{Confusion Matrix}}

  Metriks yang digunakan untuk mengukur kebaikan model classification,
  terdiri dari \texttt{accuracy}, \texttt{recall}, \texttt{specificity},
  dan \texttt{precision}.
\item
  \textbf{\emph{ROC (Receiver Operating Characteristic)}}

  Kurva yang menggambarkan performa model klasifikasi untuk seluruh
  \texttt{threshold}.
\item
  \textbf{\emph{AUC}}

  Luas area di bawah kurva \texttt{ROC}, menggambarkan keberhasilan
  model klasifikasi dalam memprediksi/membedakan kedua kelas dari
  \texttt{target\ variable}.
\item
  \textbf{\emph{Sum squared error (SSE)}}

  Jumlah dari error kuadrat. Ukuran yang bisa digunakan untuk mengukur
  kebaikan model.
\item
  \textbf{\emph{Mean squared error (MSE)}}

  Rata-rata dari error kuadrat. Ukuran yang bisa digunakan untuk
  mengukur kebaikan model.
\item
  \textbf{\emph{Root Mean squared error (RMSE)}}

  Akar kuadrat dari \texttt{MSE}.
\item
  \textbf{\emph{Mean absolute error (MAE)}}

  Rata-rata dari absolut error. Ukuran yang bisa digunakan untuk
  mengukur kebaikan model.
\end{itemize}

\end{document}
